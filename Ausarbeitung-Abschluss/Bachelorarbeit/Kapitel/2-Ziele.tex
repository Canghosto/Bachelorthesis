% -----------------------------------------------------------------------
% -----------------------------------------------------------------------
% -----------------------------------------------------------------------
% Ziele
% -----------------------------------------------------------------------
% -----------------------------------------------------------------------
% -----------------------------------------------------------------------
\chapter{Ziele}
Dieser Abschnitt beschreibt die Ziele für das zu entwickelnde System und welche Anforderungen erfüllt werden sollen. 

\section{Entwicklung}
Das Hauptziel dieser Bachelorarbeit wird darin bestehen zwei lauffähige Datenhandschuhe zu entwickeln, die für die Verwendung als Eingabegerät dienen sollen. Dieser wird aus verschiedenen Sensortypen entwickelt. Diesbezüglich wird der Prozess und die Entwicklung des Datenhandschuh beschrieben, dabei werden Entscheidungen erklärt und erläutert, aus welchen Gründen die bestimmten Komponenten verwendet werden und auch den Allgemeinen Aufbau beschrieben. 

\section{Modularität}
Eine große Schwierigkeit von Handschuhen allgemein ist die Anpassung an verschiedenen Handgrößen. Um eine stabilere Befestigung und einfache Größenanpassung zu entwickeln wird sich das Projekt auch stark an die Modularität beschäftigen. 
Es soll im allgemeinen besser auf der Hand sitzen und für einen Längeren gebrauch komfortabeler ausgestattet sein, dabei wird auch Wartbarkeit des Handschuhes mit berücksichtigt.

\section{Drahtlose Verwendung}
Nutzer sollen in der Lage sein auch ohne Verwendung von Kabeln Daten zu senden, um eine größere Reichweite und Freiheit auszuüben. Die motorischen Einschränkungen sollen durch die Nutzung von der drahtlosen Verbindung minimiert werden.

\section{Verschiedene Sensormodelle}
Um nicht nur mit einem speziellem Sensor zu arbeiten und keine Vergleiche ziehen zu können, werden in dieser Arbeit mehrere Sensormodelle verwendet, um den gleichen Nutzen genauer zu betrachten. Die Achtung liegt auch daran, wie die Kommunikation des Prozessors auf solch eine Belastung Auswirkungen hat. Der erste Prototyp von Paul Bienkowski und Carolin Konietzny verwendete einen einzigen Sensormodell mit dem Ziel, ein Tastatureingabegerät zu entwickeln, der mithilfe des Datenhandschuhes simuliert werden kann.
Diese Arbeit wird sich mit den verschiedenen Sensormodellen beschäftigen, um so Einblick in verschiedene Modelle zu bekommen.